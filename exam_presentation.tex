\documentclass{beamer}
\usepackage{graphics} %required to insert images 
\usepackage{listings}
\usepackage{siunitx}

\usetheme{AnnArbor}
\usecolortheme{beaver}


\title{Snow coverage variation in Gran Paradiso National Park}
\subtitle{Comparing data from February 2016 and 2023}


\author{Gaia Benevenga}

\date{January 2024}

\begin{document}

\maketitle

\AtBeginSection[] % Do nothing for \section*
{
\begin{frame}
\frametitle{Outline}
\tableofcontents[currentsection]
\end{frame}
}

\section{Aim of the project}
\begin{frame}{About the project}
\begin{itemize}
    \item Compare snow coverage in the area of Gran Paradiso National Park using data taken from Sentinel 2 - Copernicus. 
    \item Comparison done by taking in account the \textbf{NDSI} (Normalized Difference Snow Index)  
    \item 3 clusters to identify: \\
    - snow presence\\
    - snow absence/mixture\\
    - vegetation\\
     \end{itemize}  
\end{frame}

\begin{frame}
    
\begin{figure}[Italy from NASA Visible Earth]
    \centering
    \includegraphics[width=0.7\textwidth]{Italy.jpg}
    \caption{Italy from NASA Visible Earth}
    \label{Italy from NASA visible Earth}
\end{figure}



\end{frame}


\section{Analysis in R}

\begin{frame}{First Passages}
    \begin{itemize}
        \item Data taken from Sentinel 2 - Open Hub
        \item Classification of images based on years and bands (b2, b3, b4, b8, b11)
        \item Change in resolution and dimension of b8 and b11 images using the function 'disagg'
        \item Computation of NDSI (2016 and 2023) and its difference to see variation of snow
        \item Histograms on NDSI and its variation
        
      
    \end{itemize}
\end{frame}


\begin{frame}{What is the NDSI?}
The NDSI (Normalized Difference Snow index) is designed to highlight snow-covered areas in remote sensing imagery. \\
It is calculated as the difference between the reflectance in the visible (from the green band) and shortwave infrared (SWIR) band, normalized by their sum.
It ranges from -1 to 1:  
\begin{itemize}
    \item \textbf{Positive values} (\textgreater 0.4) = snow-covered areas
    \item \textbf{Values near 0} = non-snow-covered areas or areas with mixture of snow and other features
    \item \textbf{Negative values} = abundant vegetation
\end{itemize}

The formula used is the following:\\
         \begin{equation}
         NDSI = \frac{green - SWIR}{green + SWIR}
           \end{equation}
     
\end{frame}

\begin{frame}{Results from NDSI}
    
\begin{figure}[]
    \includegraphics[width=0.4\textwidth]{carina.jpeg}
    \includegraphics[width=0.4\textwidth]{carina 2.jpeg}
\end{figure}

\end{frame}

\begin{frame}
    
\begin{figure}
    \centering
    \includegraphics[width=0.7\textwidth]{NDSIdifff.jpeg}
    \caption{Differences from 2016 and 2023\\
    \tiny - brown areas = positive difference \\ 
    \tiny - white areas = no significant variations\\
    \tiny - blue areas = negative difference }

\end{figure}

\end{frame}


\begin{frame}{The Histograms}
\begin{figure}
  
    \includegraphics[width=0.3\linewidth]{hist2016.jpeg} \hfill
    \includegraphics[width=0.3\linewidth]{hist2023.jpeg} \hfill
    \includegraphics[width=0.3\linewidth]{histdifnew.jpeg}  \hfill  
    \caption{NDSI histograms}

\end{figure}


\end{frame}



\begin{frame}{Following passages}
Based on the values that NDSI can assume, I established 3 thresholds in R in order to create \textbf{3 clusters}. \\
Then I computed the \textbf{percentages} of the 3 clusters.\\
\vspace{0.2cm}
At the end I calculated the \textbf{standard deviation} with the function 'focal' and I obtained very low values (lack of variability or high homogeneity within moving windows).  
    

\end{frame}

\begin{frame}
    
\begin{figure}
    \centering
    \includegraphics[width=0.4\textwidth]{2016.jpeg}
    \includegraphics[width=0.4\textwidth]{2023.jpeg}
    \caption{Images based on the values of the images points\\
    \tiny - 3th cluster(green points): values \textgreater  0.4 (snow)\\
    \tiny - 2th cluster (yellow points): values between 0 and 0.4 (mixture) \\
    \tiny - 0th cluster (white points): negative values between -1 and 0 (vegetation)  }
  
\end{figure}

\end{frame}


\begin{frame}[fragile]{The Graphs}
    Then I extracted the total number of pixels to compute the percentages of pixels of the 3 clusters. \\
    The function used to create the graphs is based on the
    \texttt{ggplot2} package and looks like this:\\

\begin{lstlisting}[language=R, basicstyle=\tiny]

class <- c("SNOW", "MIXTURE", "VEGETATION")
y2016 <- c(65.5, 8.5, 26)
y2023 <- c(46.5, 31, 22.5)

tabout <- data.frame(class, y2016, y2023)

p1 <- ggplot(tabout, aes(x=class, y=y2016, color=class)) + 
      + geom_bar(stat="identity", fill="white") + ylim(c(0,100))
p2 <- ggplot(tabout, aes(x=class, y=y2023, color=class)) + 
      + geom_bar(stat="identity", fill="white") + ylim(c(0,100))

plot(p1)
plot(p2)

\end{lstlisting}

\end{frame}
  

\begin{frame}{The graphs}
\begin{figure}
    \centering
    \includegraphics[width=0.4\textwidth]{GRAPH 1.jpeg} 
    \includegraphics[width=0.4\textwidth]{GRAPH 2.jpeg}
    \caption{Graphs based on percentages of snow, no-snow (mixture), vegetation\\
    \small  For 2016: snow \SI{65.5}{\percent}, mixture \SI{8.5}{\percent}, vegetation \SI{26}{\percent}\\
    \small For 2023: snow \SI{46.5}{\percent}, mixture \SI{31}{\percent}, vegetation \SI{22.6}{\percent} }
  
\end{figure}

\end{frame}

\section{Materials and Methods}
\begin{frame}{Materials and Methods}
\begin{itemize}
    \item Data collection: Copernicus Open Access Hub - Sentinel 2
    \item R packages: 'terra' 'imageRy' 'ncdf4' 'ggplot2'
    \item Code availability: Github/gaiabene/R\textunderscore code\textunderscore exam.R
    \end{itemize}

\end{frame}

\section{Final considerations}
\begin{frame}{Final considerations}

The analysis have highlighted a \textbf{variation} in snow coverage between the year 2016 and the year 2023.\\
In particular we can see how the NDSI values towards 0 (representing the absence of snow) increase but not the negative ones which instead indicate the presence of vegetation.\\
\vspace{0.5cm}
More in-depth analysis, that take in account other features like elevation, could evaluate the hypothesis of a decrease in precipitation or an early melting due to the rise in temperatures. 
    
\end{frame}
    
\end{frame}

\section{Bibliography}

\begin{frame}{References}
  \begin{itemize}
      \item Richiardi, Blonda (2021)
      "A Revised Snow Cover Algorithm to Improve Discrimination between Snow and Clouds: A Case Study in Gran Paradiso National Park"
      \item Dedieu, Carlson (2016) "On the Importance of High-Resolution Time Series of Optical Imagery for Quantifying the Effects of Snow Cover Duration on Alpine Plant Habitat"
      \item Poussin, Guigoz (2019) "Snow Cover Evolution in the Gran Paradiso National Park, Italian Alps, Using the Earth Observation Data Cube"
      \item https://eos.com
      \item https://nsidc.org      
      \end{itemize}
\end{frame}


\end{document}
